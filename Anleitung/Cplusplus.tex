\documentclass[12pt,twoside,a4paper]{article}
\usepackage{listings}
\begin{document}
\section{Hello world}
\"Offnen Sie eine Konsole auf dem Linux-Rechner. \"Offnen Sie nun mit einem Editor Ihrer Wahl ein Dokument mit dem Namen 'helloworld.cc', z.B. mit:
 \begin{lstlisting}
emacs helloworld.cc
 \end{lstlisting}
Implementieren Sie den folgenden Code:
\begin{lstlisting}
#include<iostream>

int main()
{
  std::cout << "Hello world!" << std::endl;
}
\end{lstlisting}
Kompilieren Sie das Programm mit:
\begin{lstlisting}
g++ helloworld.cc -o helloworld
\end{lstlisting}
Durch diesen Befehl wird eine ausf\"uhrbare Datei mit dem Namen 'helloworld' erstellt. F\"uhren Sie diese Datei aus:
\begin{lstlisting}
./helloworld
\end{lstlisting}

\section{Funktionen}
F\"ugen Sie Ihrem Programm nun eine Funktion hinzu:
\begin{lstlisting}
#include<iostream>

void PrintHello()
{
  std::cout << "Hello world!" << std::endl;
}

int main()
{
  PrintHello();
}
\end{lstlisting}
Kompilieren Sie erneut und f\"uhren Sie das Programm aus.\\
Sofern eine Funktion einen Wert ausgeben soll muss die \textit{void}-Anweisung mit dem Typ \textbf{return} ersetzt werden, z.B. \textbf{int} oder \textbf{double}. In diesem Fall muss sichergestellt werden, dass die Funktion eine Anweisung \textbf{return x} besitzt. Im Folgenden finden Sie eine Funktion, die zwei Integers addiers:
\begin{lstlisting}
int sum(int a, int b)
{
  int result = a + b;
  return result;
}
\end{lstlisting}
Implementieren Sie diese Funktion in Ihr Programm und lassen Sie sich das Ergebnis ausgeben.

\section{For-Schleifen}
Erweitern Sie Ihre PrintHello()-Funktion wie folgt:
\begin{lstlisting}
void PrintHello(int n)
{
  for(int i=1; i<n; i++)
    {
      std::cout << i << " Hello world!" << std::endl;
    }
}
\end{lstlisting}
Die Ausgabe "Hello world!" wird nun n-1 Mal wiederholt. Durch die Ausgabe des Index i sehen Sie die Anzahl der Wiederholungen. F\"uhren Sie die Funktion so aus, dass "Hello world!" zehn Mal ausgeben wird.

\section{if-Bedingungen}
Eine if-Bedingung erlaubt Ihnen innerhalb Ihres Codes zu kontrollieren, ob das Programm einen Abschnitt des Codes betritt oder nicht -- je nachdem, ob die Bedingung erf\"ullt (\textbf{true}) ist oder nicht (\textbf{false}). Die Syntax sieht wie folgt aus:
\begin{lstlisting}
if(Bedingung){
  .. code ..
} else if(eine andere Bedingung){
  .. weiterer Code ..
} else{
  .. code, der nur ausgefuehrt wird, wenn die vorherigen 
  Bedingungen false sind ..
}
\end{lstlisting}
Im Folgenden sind Beispiele mit verschiedenen Operatoren:
\begin{lstlisting}
if(x>a) ...groesser als
if(x<a) ...kleiner als
if(x>=a) ...groesser gleich als
if(x<=a) ...kleiner gleich als
if(x==a) ...ist gleich
if(x!=a) ...ist nicht gleich
\end{lstlisting}
Implementieren Sie in Ihrem Programm einen Integer, der dem Ergebnis Ihrer sum()-Funktion entspricht. Fragen Sie daraufhin ab, ob das Ergebnis gr\"o{\ss}er als 10 oder kleiner gleich 10 ist:
\begin{lstlisting}
int output = sum(a,b); // replace a and b with values of your choice
if(output>10) std::cout << "The sesult is greater than 10" << std::endl;
else{ std::cout << "The result is less than 10 or equal 10" << std::endl;}
\end{lstlisting}


\end{document}