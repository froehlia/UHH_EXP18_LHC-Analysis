\documentclass[12pt,twoside,a4paper]{article}
\usepackage{listings}
\begin{document}

\section*{Event Displays mit Fireworks}
Installieren Sie Fireworks und \"offnen Sie anschlie{\ss}end das $t\bar{t}$-Sample:
\begin{lstlisting}
wget http://cern.ch/cmsshow/cmsShow-8.0-2.linux.tar.gz
tar xzf cmsShow-8.0-2.linux.tar.gz
cd cmsShow-8.0-2
wget https://www.desy.de/~mstoev/teaching/Top_WQS/ttbar.root
./cmsShow ttbar.root
\end{lstlisting}
Das Sample beinhaltet 400 simulierte $t\bar{t}$-Ereignisse. Machen Sie sich mit der Software vertraut:
\begin{itemize}
\item betrachten Sie verschiedene Ereignisse
\item betrachten Sie lediglich Jets, Myonen sowie Elektronen
\item betrachten Sie Ereignisse in der Rho-Z-Achse sowie im 3D-Tower.
\end{itemize}
Versuchen Sie nun jeweils ein Ereignis f\"ur den voll-hadronischen, semi-leptonen und di-leptonen Zerfallskanal zu finden. Benutzen Sie daraufhin die Filter-Funktion. Passen Sie diese so an, dass m\"oglichst ausschlie{\ss}lich $t\bar{t}$-Ereignisse im semi-leptonischen Zerfallskanal betrachtet werden. Speichern Sie (mindestens) ein Event-Display f\"ur Ihre Auswertung!\\
\\
Betrachten Sie nun Event-Displays aus einem Single-Top-Sample:
\begin{lstlisting}
wget https://www.desy.de/~mstoev/teaching/Top_WQS/singletop.root
./cmsShow wjets.root
\end{lstlisting}
K\"onnen Sie Unterschiede feststellen, die Ihnen bei der Selektion von $t\bar{t}$-Ereignissen helfen?\\
\\
Speichern Sie f\"ur Ihre Auswertung ein Event-Display mit einem Single-Top-Ereignis. 

%\section{Hello world}
%\"Offnen Sie eine Konsole auf dem Linux-Rechner. \"Offnen Sie nun mit einem Editor Ihrer Wahl ein Dokument mit dem Namen 'helloworld.cc', z.B. mit:
% \begin{lstlisting}
%emacs helloworld.cc
% \end{lstlisting}


\end{document}