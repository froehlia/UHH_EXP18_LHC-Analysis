\subsection{Aufgabe 3: Rekonstruktion der Top-Quark-Masse}
\label{topreco}

Im letzten Teil des Versuches werden Sie die Top-Quark-Masse rekonstruieren. Hierf"ur werden, wie auch schon im vorherigen Teil, Top-Antitop-Ereinisse verwendet. Zun"achst wird Ihnen hier die Methode vorgestellt, im Anschluss legen Sie selbst Hand an und optimieren eine bereits in Grundz"ugen vorhandene Rekonstruktion.

\subsubsection{Methode zur Rekonstruktion}
\label{subsec:topreco_intro}

Um die Top-Quark-Masse zu rekonstruieren, werden wir, wie gehabt, Ereignisse mit zwei Top Quarks verwenden. Dabei nehmen wir an, dass dieses System von Top Quarks semi-leptonisch zerf"allt und ein Myon und fehlende transversale Energie (hervorgerufen durch das nicht-detektierte Neutrino) sowie Jets im Endzustand hervorruft. Das System aus beiden Top Quarks soll rekonstruiert und der Mittelwert beider rekonstruierter Top-Quark-Massen verwendet werden. Die Rekonstruktion verl"auft in drei Schritten, die im Folgenden erl"autert werden.

\begin{enumerate}
\item \textbf{Rekonstruktion des Neutrinos:} Unter der Annahme, dass das Neutrino die einzige Quelle fehlender transversaler Energie ist, und dass Myon und Neutrino von einem realen W-Boson erzeugt wurden, kann eine Formel f"ur die z-Komponente des Neutrino-Viererimpulses gefunden werden. Die transversalen Komponenten erh"alt man direkt aus der Messung der fehlenden Transversalenergie und die Masse verschwindet im Standardmodell. 

\textbf{Aufgabe:} Machen Sie sich klar, dass aus der Annahme von 

\begin{equation}
\left(P_{\text{W}}\right)^2 = \left(P_{\mu}\right)^2 + \left(P_{\nu}\right)^2,
\end{equation}

\noindent wobei $P$ den Viererimpuls eines gegebenen Teilches darstellt, und 

\begin{equation}
E_{\text{T}}^{\text{miss}} = p_{\text{T, }\nu},
\end{equation}

\noindent wobei $E_{\text{T}}^{\text{miss}}$ die fehlende Transversalenergie bezeichnet, die z-Komponente des Neutrino-Viererimpulses gegeben ist durch: 

\begin{equation}
p_{z,\nu}^\pm = \frac{\mu\,p_{z,\mu}}{p_{T,l}^2} \pm \sqrt{\frac{\mu^2p_{z,\mu}^2}{p_{T,l}^4}-\frac{E_\mu^2\,p_{T,\nu}^2-\mu^2}{p_{T,\mu}^2}},
\end{equation}

\noindent wobei $\mu = M_W^2 / 2 + p_{T,\mu}\,p_{T,\nu}\,\text{cos}\Delta\phi$ und $p_{\mu}$ und $p_{\nu}$ die Dreierimpulse von Muon und Neutrino kennzeichnet.

Diese Gleichung kann 0, 1 oder 2 reelle L"osungen haben. Falls keine reelle L"osung existiert, wird stattdessen der Realteil der komplexen verwendet. Im Falle von 2 L"osungen wird das im Folgenden beschriebene Procedere f"ur jede L"osung wiederholt.

Damit ist der gesamte Viererimpuls des Neutrinos rekonstruiert und wir fahren mit dem zweiten Schritt fort:

\item \textbf{Rekonstruktion des Top-Antitop-Quark Systems:} Wir nehmen an, dass die Quarks aus dem Top Quark Zerfall (ein b-Quark aus dem leptonischen, drei Quarks, wovon eines ein b-Quark ist, aus dem hadronischen Top Quark) als Jets von Detektor registriert werden. Die einfachste Annahme w"are also, dass wir die ''richtigen'' vier Jets finden m"uessen, um alle Teilchen beider Zerf"alle zu identifizieren. 

Es kann jedoch vorkommen, dass zwei Quarks in einer sehr "ahnlichen Richtung produziert werden und deshalb auch nur als ein einziger Jet im Detektor gemessen werden. Ebenso kann (z.B. durch Gluon-Abstahlung im Endzustand) ein zus"atzlicher Jet produziert werden, der f"ur die Rekonstruktion der beiden Top Quarks ber"ucksichtigt werden muss. Wir betrachten also alle F"alle, in denen die Top Quarks aus mindestens 3, h"ochstens jedoch 5 Jets rekonstruiert werden k"onnen. Der Einfachheit halber nehmen wir an, dass immer ein Jet vom leptonischen Top Quark erzeugt wird und zwei bis vier vom hadronischen. 

Nun zur eigentlichen Arbeit: Das Ereignis habe $N$ Jets. Dann werden $N$ leptonische Top Quark Kandidaten  gebildet, jeweils aus dem Viererimpuls des Muons, des Neutinos und eines Jets:

\begin{equation}
P_{\text{t}}^{\text{lep}} = P_{\mu} + P_{\nu} + P_{\text{jet}}.
\end{equation}

F"ur jeden leptonischen Kandidaten verbleiben $N-1$ Jets, die f"ur das hadronische Top Quark zur Verf"ugung stehen. Wie oben diskutiert werden daraus alle m"oglichen Permutationen von 3, 4 oder 5 Jets gebildet und zum hadronischen Top Quark Kandidaten erkl"art:

\begin{equation}
P_{\text{t}}^{\text{had}} = \sum_{\text{i}} P_{\text{jet, i}},
\end{equation}

\noindent wobei die Summe "uber alle Jets in der betrachteten Permutation l"auft. F"ur jeden leptonischen Kandidaten wird also eine Vielzahl hadronischer Kandidaten gebildet.  

Au"serdem machen wir uns das Vorkommen von Jets mit b-Tag zunutze. Wenn mindestens ein Jet als b-Jet identifieziert wurde, muss er f"ur eines von beiden Top Quarks verwendet werden. Sollten mindestens zwei b-Jets vorhanden sein, muss jedes Top Quark einen enthalten. Alle Kandidatenpaare, die diese Bedingung nicht erf"ullen, werden verworfen. Im letzten Schritt wird aus allen verbliebenen Kandidatenpaaren das beste ausgew"ahlt.

\item \textbf{Auswahl des besten Paares:} Wir w"ahlen das Paar als unsere finalen rekonstruierten Top Quarks aus, bei dem die Differenz der rekonstruierten Massen minimal ist - wie es bei einem tats"achlichen Paar von Top Quarks der Fall sein sollte. Falls die Massendifferenz 10\,GeV "ubersteigt, wird die Paarung verworfen. Die rekonstruierte Top Quark Masse ergibt sich aus dem arithmetischen Mittel der beiden individuellen rekonstruierten Massen:

\begin{equation}
M_{\text{t}}^{\text{rec}} = \frac{M_{\text{t}}^{\text{lep}} + M_{\text{t}}^{\text{had}}}{2}.
\label{eq:mtoprec}
\end{equation}


\end{enumerate}

\subsubsection{Messung der rekonstruierten Top Quark Masse}
Nun sollen Sie die obigen Schritte in der Praxis anwenden und optimieren. Die Analysesoftware enth"alt bereits alle n"otigen Funktionen, um die Schritte 1 - 3 aus dem vorigen Kapitel durchzuf"uhren. 

\begin{itemize}
\item Verifizieren Sie, dass die Funktionen MyAnalysis::ReconstructTTbar und MyAnalysis::SelectBestTTbarHypothesis die Schritte 1-2 und 3 repr"asentieren und verstehen Sie den Aufbau der Funktionen. 

\item Fordern Sie in einer neuen Selektion mindestens 3 Jets in jeden Ereignis. Setzen Sie die Variable ''m\_top\_avg'' gem"a"s der in Gleichung \ref{eq:mtoprec} gegebenen Definition. Falls Sie Erfolg haben, wird das letzte Histogramm in der Liste nun mit dieser Variable gef"ullt werden.

\item Optimiern Sie diese Selektion, bei der Sie gr"o"stenteils Top-Antitop-Ereignisse selektieren, ohne unn"otig viele Ereignisse zu verwerfen. \textit{Hinweis: Sie sollten nicht mehr als eine oder zwei andere Variablen (zus"atzlich zur Anzahl an Jets) verwenden. "Uberlegen Sie, welche Variablen f"ur die im vorigen Kapitel beschriebene Rekonstruktion hilfreich war!}

\item Modifizieren Sie die Art der Rekonstruktion, indem Sie die Mindestzahl der im hadronischen Top Quark verwendeten Jets variieren (ganzzahlige Werte zwischen 2 und 4). Hierf"ur k"onnen sie das Argument der Funktion MyAnalysis::ReconstructTTbar verwenden, das im Moment auf 3 eingestellt ist. Bei welcher Mindestanzahl erscheint der Peak bei der Top Quark Masse am sch"arfsten? Was k"onnte Grund daf"ur sein?

\item \textbf{BONUS:} Sie k"onnen ebenso die maximal zul"assige Massendifferenz zwischen hadronischer und leptonischer rekonstruierter Top Quark Masse variieren. Finden Sie heraus, wie, und probieren Sie unterschiedliche Werte aus! Was beobachten Sie? K"onnen Sie einen besseren Wert als 10\,GeV finden?

\item Legen Sie sich auf eine Kombination aus Selektion und Mindestanzahl an Jets fest, die Sie im Folgenden verwenden. Erl"autern Sie Ihre Wahl

\end{itemize}

Ihr gemessener Wert der Top Quark Masse wird in diesem Versuch ermittelt, indem eine Gau"skurve an die Verteilung der rekonstruierten Top Quark Masse gefittet wird. Der Mittelwert der Kurve entspricht der Masse, die Unsicherheit auf den Mittelwert entspricht der statistischen Unsicherheit Ihrer Messung. Wie auch bei der Messung des Wirkungsquerschnitts betrachten wir die Unsicherheit auf JEC als systematische Unsicherheit. 

F"ur die Messung wird die Verteilung in Daten abz"uglich der Untergr"unde verwendet. Dies wurde bereits f"ur Sie implementiert.

\begin{itemize}
\item Modifizieren Sie den Code in der Datei example.C und aktivieren Sie den Fit! Experimentieren Sie mit der oberen und unteren Grenze des Fits, beachten Sie jedoch, dass stets nur der zentrale Teil der Verteilung gefittet werden sollte.

\item Ermitteln Sie den Wert und die statistische Unsicherheit der mit Ihrer Selektion und Methode (s.o.) ermittelten rekonstruierten Top Quark Masse.

\item Wiederholen Sie die Selektion und den Fit f"ur eine variierte JEC (up, down). Die gr"o"sere Differenz von der im vorigen Schritt ermittelten Masse entspricht der systematischen Unsicherheit Ihrer Messung. 

\item Vergleichen Sie Ihr Ergebnis mit dem zum Zeitpunkt der Versuchsdurchf"uhrung aktuellen weltweiten Mittelwertes der Top Quark Masse!
\end{itemize}